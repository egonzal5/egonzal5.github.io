\documentclass[11pt]{article}
\usepackage{amsfonts,amsmath}
\usepackage{verbatim}
\usepackage{booktabs}
\usepackage{enumitem}

\DeclareMathOperator*{\argmin}{arg\,min}
\DeclareMathOperator*{\argmax}{arg\,max}

\setlength{\textwidth}{6in}
\setlength{\oddsidemargin}{0.0in}
\setlength{\evensidemargin}{0.0in}
\setlength{\topmargin}{-0.5in}
\setlength{\textheight}{46\baselineskip}
\title{\vspace{-0.7in} CMSC320 -- Introduction to Data Science\\
\textbf{Final Tutorial}}

\author{Fall Semester 2019\\[.2in]
  John Dickerson\\
  University of Maryland\\
  Department of Computer Science}

\date{}

\newtheorem{definition}{Definition}[section]
\newtheorem{axiom}{Axiom}[section]
\newtheorem{lemma}{Lemma}[section]
\newtheorem{theorem}{Theorem}[section]
\newtheorem{corollary}{Corollary}[section]
\newtheorem{algorithm}{Algorithm}[section]

\pagenumbering{arabic}

\begin{document}

\maketitle
%\thispagestyle{empty}
%\pagestyle{empty}

%\vspace{1in}

\section{Summary}
In lieu of a final exam, CMSC320 students will turn in a tutorial that will walk users through the entire data science pipeline: data curation, parsing, and management; exploratory data analysis; hypothesis testing and machine learning to provide analysis; and then the curation of a message or messages covering insights learned during the tutorial.  Students may choose an application area and dataset(s) that are of interest to them; please feel free to be creative about this!  (For some ideas and possible data sources, see the slides from the first lecture.)  The tutorial should be self-contained, a mix of Markdown prose and Python code, and delivered as a GitHub statically-hosted Page (described below).

As example tutorials, check out:
\begin{itemize}
\item The golden age of rap: \texttt{http://rstumbaugh.me/hiphop-analysis/}
\item Predicting a win in Rainbow Six: Siege: \texttt{https://jiglesia3.github.io/}
\item What makes the best defensive footballers?  \texttt{https://bdaisey.github.io/}
\item Maryland and peer institutions' faculty/student counts: \texttt{https://krixly.github.io/}.
\item Analysis of crime data in College Park: \texttt{https://andresgogo.github.io/}
\end{itemize}
In general, the tutorial should contain at least 1500 words of prose and 150 lines of (non-padded, legitimate) Python code, along with appropriate documentation, visualization, and links to any external information that might help the reader.  You are welcome to do this project individually or in a group of size at most three; we'll scale up the expectations accordingly as group size increases.

\subsection{Github Pages}
GitHub provides a service called Pages (\texttt{https://pages.github.com/}) that provides website hosting functionality backed by a GitHub-based git repository.  We would like you to host your final project on a GitHub Pages project site.  To do this, you will need to:
\begin{enumerate}
\item Create a GitHub account (or use the one you already have) with username \texttt{username}.
\item Create a git repository titled \texttt{username.github.io}; make sure \texttt{username} is the same as whatever you chose for your global GitHub account.
\item Create a project within this repository.  This is where you'll dump your iPython Notebook file and an HTML export of that Notebook file.  
\end{enumerate}
These instructions are also given directly on the front page of \texttt{https://pages.github.com/}; following those instructions should be fine!

The deliverable to the CMSC320 staff will then be a single URL pointing to this publicly-hosted GitHub Pages-backed website.  It is due by the CMSC320 university-wide pre-scheduled date of \textbf{4:00PM on Monday, December 16th}.  We will not (\emph{cannot}) accept late assignments.

\noindent\textbf{Please make sure to include your name (and the names of all group members) at the top of your deliverable, after the title.}

\section{Grading}
We will assign a numeric score between $1$ and $10$ for each of the following six dimensions:
\begin{enumerate}
\item \textbf{Motivation.}  Does the tutorial make the reader believe the topic is relevant or important (i) in general and (ii) with respect to data science?
\item \textbf{Understanding.}  After reading through the tutorial, does an uninformed reader feel informed about the topic?  Would a reader who already knew about the topic feel like s/he learned more about it?
\item \textbf{Other resources.}  Does the tutorial link out to other resources (on the web, in books, etc) that would give a lagging reader additional help on specific topics, or an advanced reader the ability to dive more deeply into a specific application area or technique?
\item \textbf{Prose.}  Does the prose portion of the tutorial actually add to the content of the deliverable?
\item \textbf{Code.}  Is the code well written, well documented, reproducible, and does it help the reader understand the tutorial?  Does it give good examples of specific techniques?
\item \textbf{Subjective evaluation.}  If somebody linked to this tutorial from, say, Hacker News, would people actually read through the entire thing?
\end{enumerate}

\begin{center}{\Large
\begin{tabular}{ccc}
\toprule
Dimension & Points Received & Points Possible\\\midrule
Motivation & & 10\\
Understanding & & 10\\
Further Resources & & 10\\
Prose & & 10\\
Code & & 10\\
Subjective Evaluation & & 10\\\midrule
Total & & 60\\\bottomrule
\end{tabular}%
}
\end{center}

\end{document}
