\documentclass{beamer}

\usepackage{amssymb}
\usepackage{fancyvrb}
\usepackage{stmaryrd}
\usepackage{graphicx}
\usefonttheme{serif}


\newcommand{\Nat}{\mathbb{N}}

\title{Data Science}%\texorpdfstring{$\mathbb{N}$}}
\subtitle{\blueit{Midterm Review}}
\date{March 29\textsuperscript{th}, 2021}

\usetheme{jmct}

\usepackage{calc}

\newcommand{\textover}[3][l]{%
 % #1 is the alignment, default l
 % #2 is the text to be printed
 % #3 is the text for setting the width
 \makebox[\widthof{#3}][#1]{#2}%
 }

\newcommand{\blueit}[1]{%
  {\color{dark-lucid-blue}#1}%
}
\newcommand{\blueite}[1]{%
  \blueit{\emph{#1}}%
}


\newcommand{\myquote}[3]{
  ``#1''
  \vspace{3pt}
  \hrule
  \begin{flushright}
  --- \blueit{\emph{#2}}, \emph{#3}
  \end{flushright}
}

\begin{document}
	\frame {
		\titlepage
	}

%%%%%%%%%%%%%%%%%%%%%%%%%%%%%%%%%%%%%%%% 
%%% Intro
%%%%%%%%%%%%%%%%%%%%%%%%%%%%%%%%%%%%%%%% 

  \frame{
    \frametitle{Before we start...}
      \begin{enumerate}
        \item<2 -> Our mod of the day.
        \item<3 -> Project 2
        \item<4 -> Equation(s) you were promised
        \item<5 -> How this lecture is going to work.
      \end{enumerate}
  }

  \frame{
    \frametitle{Our moderator}
      \begin{enumerate}
        \item<2 -> Laura!
      \end{enumerate}
  }


  \frame{
    \frametitle{Project 2}
      \begin{enumerate}
        \item<2 -> Well done, submission rate was much higher than Project 1.
      \end{enumerate}
  }

  \begin{frame}
    \frametitle{An equation you were promised:}
    This was written wrong:
    \begin{equation*}
      (1 + \frac{\lambda}{N})^{-2}
    \end{equation*}
  \end{frame}

  \begin{frame}
    \frametitle{An equation you were promised:}
    Should have been:
    \begin{equation*}
      \sqrt{1 + \frac{\lambda}{N}}
    \end{equation*}
  \end{frame}

  \begin{frame}
    \frametitle{An equation you were promised:}
    Let's say you had a variable where half the data was missing
    ($\lambda = 0.5$) and you used $N = 5$ for the number of generated data
    sets:
    \begin{equation*}
      \sqrt{1 + \frac{0.5}{5}} = 1.049
    \end{equation*}
  \end{frame}

  \begin{frame}
    \frametitle{An equation you were promised:}
    How much better would it be if you used an `infinite' number of generated
    data sets?
    \begin{equation*}
      \sqrt{1 + \frac{0.5}{5}} = 1.0
    \end{equation*}
  \end{frame}

  \begin{frame}
    \frametitle{To a notebook!}
  \end{frame}

  \begin{frame}
    \frametitle{Another equation you were promised:}
    Pooled Slope Estimate: One way to `average' a pooled analysis (multiple imputation)
    \begin{enumerate}
      \item<2-> Calculate the linear regression for each imputed data-set
      \item<3-> Each linear regression has its own slope: $\beta_{1i}$
      \item<4-> Just average them!
    \end{enumerate}
    \onslide<5 ->{
    \begin{equation*}
      \beta_{1p} = \frac{\beta_{11} + \dots + \beta_{1n}}{n}
    \end{equation*}}
  \end{frame}

  \begin{frame}
    \frametitle{Another equation you were promised:}
    That was pretty na\"{i}ve, let's try something more sophisticated:
    \begin{enumerate}
      \item<2-> Calculate the linear regression for each imputed data-set
      \item<3-> Each linear regression has its own slope: $\beta_{1i}$
      \item<4-> Each linear regression has its own \blueit{standard error}: $Z_{i}$
      \item<5-> Do the following fancy weighted average:
    \end{enumerate}
    \onslide<6 ->{
    \begin{equation*}
      V_{s} = \frac{\sum{Z_{i}}}{n} + (1 + n^{-1}) * \frac{1}{n - 1} * \sum{(\beta_{1i} - \beta_{1p})^2}
    \end{equation*}}
  \end{frame}


  \begin{frame}
    \frametitle{How this lecture is going to work}

    \begin{enumerate}
      \item<2 -> I will be clear about what will \blueit{not} be on the exam
      \item<3 -> I will explain how the midterm will be administered.
      \item<4 -> We will discuss expectations
    \end{enumerate}
  \end{frame}

%%%%%%%%%%%%%%%%%%%%%%%%%%%%%%%%%%%%%%%% 
%%% Part 1: What's not on the exam
%%%%%%%%%%%%%%%%%%%%%%%%%%%%%%%%%%%%%%%% 

  \begin{frame}
  \centering
    {\fontsize{120}{88} Part I: What's not on the exam}
  \end{frame}

  \frame{
    \frametitle{NLTK}
      \begin{enumerate}
        \item<2 -> We didn't cover NLTK in the lectures, so don't worry about it.
        \item<3 -> Everything else (lectures \blueit{and} readings \blueit{and} projects) is fair game.
        \item<4 -> Aside: It has been brought to my attention that NLTK is not on the course docker images: I will fix this.
      \end{enumerate}
  }

  \frame{
    \frametitle{\texttt{git}}
      \begin{enumerate}
        \item<2 -> You should still learn it.
      \end{enumerate}
  }

  \frame{
    \frametitle{\texttt{docker}}
      \begin{enumerate}
        \item<2 -> I'm not evil.
      \end{enumerate}
  }

%%%%%%%%%%%%%%%%%%%%%%%%%%%%%%%%%%%%%%%% 
%%% Part 2: Administering the test.
%%%%%%%%%%%%%%%%%%%%%%%%%%%%%%%%%%%%%%%% 

  \begin{frame}
  \centering
    {\fontsize{120}{88} Part II: Administering the Exam}
  \end{frame}

  \frame{
    \frametitle{Taking the exam}
      \begin{enumerate}
        \item<2 -> At 00:00 EDT on March 31\textsuperscript{st} the exam will \blueit{go live} on ELMS
        \item<3 -> At 23:59 EDT on March 31\textsuperscript{st} the exam will \blueit{close} on ELMS
        \item<4 -> We will only answer \blueit{clarifying} questions that are asked \blueit{in private}
        \item<5 -> This is the only thing this semester where public discussion is forbidden.
        \item<6 -> Please respect these rules.
      \end{enumerate}
  }


%%%%%%%%%%%%%%%%%%%%%%%%%%%%%%%%%%%%%%%% 
%%% Part 3: Expectations
%%%%%%%%%%%%%%%%%%%%%%%%%%%%%%%%%%%%%%%% 

  \begin{frame}
  \centering
    {\fontsize{120}{88} Part III: Great Expectations}
  \end{frame}

  \frame{
    \frametitle{What to expect}
      \begin{enumerate}
        \item<2 -> Exam is designed for \blueit{approx.} 90 minutes
        \item<3 -> Feel free to typeset, our eyes will thank you
        \item<4 -> Explain. your. process.
        \item<5 -> Nothing will be accepted after 23:59 EDT on March 31\textsuperscript{st}, 2021 CE (AD).
      \end{enumerate}
  }


%%%%%%%%%%%%%%%%%%%%%%%%%%%%%%%%%%%%%%%% 
%%% Conclusion
%%%%%%%%%%%%%%%%%%%%%%%%%%%%%%%%%%%%%%%% 

  \frame{
    \frametitle{Thanks for your time!}

     :) 
  }


\end{document}
