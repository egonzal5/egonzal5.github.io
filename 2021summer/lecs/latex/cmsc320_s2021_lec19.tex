\documentclass{beamer}

\usepackage{amssymb}
\usepackage{fancyvrb}
\usepackage{stmaryrd}
\usepackage{graphicx}
\usepackage{tikz}
\usefonttheme{serif}


\newcommand{\Nat}{\mathbb{N}}

\title{Data Science}%\texorpdfstring{$\mathbb{N}$}}
\subtitle{\blueit{Introduction to Machine Learning: Preliminaries}}
\date{April 7, 2021}

\usetheme{jmct}

\usepackage{calc}

\newcommand{\textover}[3][l]{%
 % #1 is the alignment, default l
 % #2 is the text to be printed
 % #3 is the text for setting the width
 \makebox[\widthof{#3}][#1]{#2}%
 }

\newcommand{\blueit}[1]{%
  {\color{dark-lucid-blue}#1}%
}
\newcommand{\blueite}[1]{%
  \blueit{\emph{#1}}%
}
\newcommand{\redit}[1]{%
  {\color{my-magenta}#1}%
}


\newcommand{\myquote}[3]{
  ``#1''
  \vspace{3pt}
  \hrule
  \begin{flushright}
  --- \blueit{\emph{#2}}, \emph{#3}
  \end{flushright}
}

\begin{document}
	\frame {
		\titlepage
	}

\newcommand{\withwidth}[2]{%
  \makebox[\widthof{#2}][c]{#1}%
}

%%%%%%%%%%%%%%%%%%%%%%%%%%%%%%%%%%%%%%%% 
%%% Intro
%%%%%%%%%%%%%%%%%%%%%%%%%%%%%%%%%%%%%%%% 

  \frame{
    \frametitle{Recap: The Pipeline}
      \begin{center}
      \begin{tikzpicture}[vert/.style={rectangle, draw=lucid-blue
                                      ,radius=1cm
                                      ,minimum width={width("Analysis, Hypothesis Testing, \& ML")+2pt}
                                      ,font=\small}
                         ,every edge/.style={draw=my-magenta}]
        \node [vert] (a) at (0  ,5) {Data Collection};
        \node [vert] (b) at (0  ,4) {Data Processing};
        \node [vert] (c) at (0  ,3) {Exploratory Analysis};
        \node [vert] (d) at (0  ,2) {Analysis, Hypothesis Testing, \& ML};
        \node [vert] (e) at (0  ,1) {Insight \& Policy Decision};

        \draw 
              (a) edge[->] (b)
              (b) edge[->] (c)
              (c) edge[->] (d)
              (d) edge[->] (e);

      \end{tikzpicture}%
      \end{center}
  }

  \frame{
    \frametitle{What we're doing next:}
      \begin{center}
      \begin{tikzpicture}[vert/.style={rectangle, draw=lucid-blue
                                      ,radius=1cm
                                      ,minimum width={width("Analysis, Hypothesis Testing, \& ML")+2pt}
                                      ,font=\small}
                         ,every edge/.style={draw=my-magenta}]
        \node [vert] (a) at (0  ,5) {Data Collection};
        \node [vert] (b) at (0  ,4) {Data Processing};
        \node [vert] (c) at (0  ,3) {Exploratory Analysis};
        \node [vert,fill=lucid-blue,text=white] (d) at (0  ,2) {Analysis, Hypothesis Testing, \& ML};
        \node [vert] (e) at (0  ,1) {Insight \& Policy Decision};

        \draw 
              (a) edge[->] (b)
              (b) edge[->] (c)
              (c) edge[->] (d)
              (d) edge[->] (e);

      \end{tikzpicture}%
      \end{center}
  }

  \frame{
    \frametitle{Motivation}
      In previous lectures I've mentioned things like a ``linear model'', or ``statistical model'', but...
      \begin{enumerate}
        \item<2 -> We skipped how you would \blueit{\withwidth{make}{reason about}} such a model
        \item<3 -> We skipped how you would \blueit{reason about} such a model
        \item<4 -> Now that we know how to get our data in order, it's time to really get our hands dirty!
      \end{enumerate}
  }


  \frame{
    \frametitle{What Machine Learning is not}
      \onslide<2->{Objective.}
      \begin{enumerate}
        \item<3 -> \blueite{Lots} of judgement gets used
        \item<4 -> \blueite{Lots} of heuristics get applied
        \item<5 -> Anyone who tells you differently is trying to sell you something.
      \end{enumerate}
  }

%%%%%%%%%%%%%%%%%%%%%%%%%%%%%%%%%%%%%%%% 
%%% Preliminaries
%%%%%%%%%%%%%%%%%%%%%%%%%%%%%%%%%%%%%%%% 


  \frame{
    \frametitle{Let's flip a coin}
      Coins/dice are fantastic, we'll often talk about `flipping' a coin when it
      comes to reasoning about probabilities.
      \begin{enumerate}
        \item<2 -> A coin represents a \blueite{random variable}, $v$
        \item<3 -> $v$ can have one of two outcomes: \blueit{Heads} (1) and \blueit{Tails} (0)
        \item<4 -> Each $v$ has an associate \blueit{distribution} that gives the probabilities
                   of $v$ realizing each of its possible values.
      \end{enumerate}
  }

  \frame{
    \frametitle{Great Expectations}
      Each random variable also has an \blueite{expected value}
      \begin{enumerate}
        \item<2 -> What's the expected value for a coin?
        \item<3 -> A 10-sided die?
        \item<4 -> Two 6-sided die?
        \item<5 -> Notice anything?
      \end{enumerate}
  }

  \frame{
    \frametitle{Continuing with distributions}
      Cons/Dice are \blueit{discrete} distributions, but \blueit{continuous}
      distributions are also very important.

      \vspace{1cm}

      Common distributions
      \begin{enumerate}
        \item<2 -> The \blueite{\withwidth{Uniform}{Binomial}} distribution
        \begin{itemize}
          \item<2 -> Defined by an interval
        \end{itemize}
        \item<3 -> The \blueite{\withwidth{Normal}{Binomial}} distribution: $\mathcal{N}(\mu,\sigma^2)$
        \begin{itemize}
          \item<3 -> Defined by an \blueit{mean} ($\mu$), and its
                     \blueit{standard deviation} ($\sigma$)
        \end{itemize}
        \item<4 -> The \blueite{Binomial} distribution: $\mathcal{B}(n,p)$
        \begin{itemize}
          \item<3 -> Defined by an \blueit{number of yes/no trials} ($n$), and the
                     \blueit{probability of `yes'} ($p$)
        \end{itemize}
      \end{enumerate}
  }

  \frame{
    \frametitle{Potential Problem?}
    Take the uniform distribution over $[0,1]$

    \vspace{.2cm}

    Since in a continuous space there are $\infty$-many possible points, within
    this interval, the probability for any given point is $\frac{x}{\infty} \approx 0$

      \begin{enumerate}
        \item<2 -> Do we pack it up?
        \item<3 -> No, we use \blueite{calculus}!
      \end{enumerate}
  }

  \frame{
    \frametitle{The \redit{other} PDF}

    We represent a continuous distribution as a \blueite{probability density
    function} (PDF):

      \begin{enumerate}
        \item<2 -> The probability of seeing a value \redit{within} a certain interval
                   equals the \blueite{integral} of the density function over that interval
        \item<3 -> ``But I hate calculus!'', I hear you say. Okay...
      \end{enumerate}
  }

  \begin{frame}[fragile]
    \frametitle{Speaking in Uniform Code}
    We're computer scientists, let's write some code to gain an intuition about
    these things:

    \vspace{1cm}

    For the Uniform distribution:

    \vspace{0.5cm}
  
        \centering
    {\color{dark-gray}
        \begin{BVerbatim}
def uniform_pdf(x: float) -> float:
    return 1 if 0 <= x < 1 else 0
        \end{BVerbatim}
  }

  \end{frame}

  \begin{frame}[fragile]
    \frametitle{Speaking in Normal Code}
    We're computer scientists, let's write some code to gain an intuition about
    these things:

    \vspace{0.5cm}

    For the Normal distribution: To the notebook

  \end{frame}

  \frame{
    \frametitle{PDF to CDF}

    PDFs are great, but we're not always asking a question like ``How likely is $X$'',
    sometimes we want to ask is the probability of $X$ \blueite{less than $Y$}?

      \begin{enumerate}
        \item<2 -> For that we have \redit{Cumulative density functions}!
      \end{enumerate}
  }


  \begin{frame}[fragile]
    \frametitle{Speaking in Uniform Code}

    For the Uniform distribution:

    \vspace{0.5cm}
  
        \centering
    {\color{dark-gray}
        \begin{BVerbatim}
def uniform_cdf(x: float) -> float:
    if x < 0:   return 0
    elif x < 1: return x
    else:       return 1
        \end{BVerbatim}
  }
  \end{frame}


  \frame{
    \frametitle{Hypothesis Testing}

    Now that we have some intuition for PDF vs CDF, we can talk about testing a \blueite{hypothesis}.

    \vspace{0.5cm}
    \onslide<2 ->{Example hypotheses:}
      \begin{enumerate}
        \item<3 -> Is this coin fair?
        \item<4 -> Data Scientists Prefer Python
        \item<5 -> Student who take class with Prof X are more likely to be involved in violent events.
      \end{enumerate}
  }


  \frame{
    \frametitle{Hypothesis Testing}

    To be disciplined about it, we need a \blueite{Null Hypothesis}: $H_0$.

    \vspace{0.5cm}
      \begin{enumerate}
        \item<2 -> $H_0$ is the `default' position on a question
        \item<3 -> You can have multiple hypoteses $H_1, H_2 \dots$ for each null hypothesis.
      \end{enumerate}
  }

  \frame{
    \frametitle{Simple Test}

    We want to see if a coin if fair:

    \vspace{0.5cm}
      \begin{enumerate}
        \item<2 -> What's our $H_1$?
        \item<3 -> WHat's our $H_0$?
      \end{enumerate}
  }

  \frame{
    \frametitle{Simple Test}

    We want to see if a coin if fair:

    \vspace{0.5cm}
      \begin{enumerate}
        \item<2 -> Let's write some code to see.
      \end{enumerate}
  }



%%%%%%%%%%%%%%%%%%%%%%%%%%%%%%%%%%%%%%%% 
%%% Conclusion
%%%%%%%%%%%%%%%%%%%%%%%%%%%%%%%%%%%%%%%% 

  \frame{
    \frametitle{Thanks for your time!}

     :) 
  }


\end{document}
