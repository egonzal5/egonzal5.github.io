\documentclass{beamer}

\usepackage{amssymb}
\usepackage{fancyvrb}
\usepackage{stmaryrd}
\usepackage{graphicx}
\usefonttheme{serif}


\newcommand{\Nat}{\mathbb{N}}

\title{SciPy + Pandas}%\texorpdfstring{$\mathbb{N}$}}
\subtitle{Data Science\\
          (but this is a subtitle)}
\date{Febuary 15\textsuperscript{st}, 2021}

\usetheme{jmct}

\usepackage{calc}

\newcommand{\textover}[3][l]{%
 % #1 is the alignment, default l
 % #2 is the text to be printed
 % #3 is the text for setting the width
 \makebox[\widthof{#3}][#1]{#2}%
 }

\newcommand{\blueit}[1]{%
  {\color{dark-lucid-blue}#1}%
}
\newcommand{\blueite}[1]{%
  \blueit{\emph{#1}}%
}


\newcommand{\myquote}[3]{
  ``#1''
  \vspace{3pt}
  \hrule
  \begin{flushright}
  --- \blueit{\emph{#2}}, \emph{#3}
  \end{flushright}
}

\begin{document}
	\frame {
		\titlepage
	}

%%%%%%%%%%%%%%%%%%%%%%%%%%%%%%%%%%%%%%%% 
%%% Intro
%%%%%%%%%%%%%%%%%%%%%%%%%%%%%%%%%%%%%%%% 

  \frame{
    \frametitle{This Lecture}
    Scipy: Don't reinvent the wheel. Pandas: making data manageable.
  }

  \frame{
    \frametitle{Before we start...}
      \begin{enumerate}
        \item<2 -> Assignment issues
      \end{enumerate}
  }

  \frame{
    \frametitle{Assignment}
      \begin{enumerate}
        \item<2 -> Some folks are seeing an issue with lxml.
        \item<3 -> Not sure what's happening there, I am investigating.
        \item<4 -> I will communicate the solution/resolution as soon as we know.
      \end{enumerate}
  }


%%%%%%%%%%%%%%%%%%%%%%%%%%%%%%%%%%%%%%%% 
%%% Data
%%%%%%%%%%%%%%%%%%%%%%%%%%%%%%%%%%%%%%%% 

  \frame{
    \frametitle{SciPy}
    SciPy is one of the best things to come out of the Python ecosystem

      \begin{enumerate}
        \item<2 -> Lots of mathematical functions, implemented over man
                   standard python/numpy types.
        \item<3 -> Numerical integration: (scipy.integrate)
        \item<4 -> Solving optimization problems (scipy.optimize)
        \item<5 -> Lots of linear algebra
        \item<6 -> and much more!
      \end{enumerate}

  }

  \frame{
    \frametitle{What you need to know about SciPy}
    If other library authors have done their job:
    \onslide<2->{\blueit{not much}}
  }

  \frame{
    \frametitle{What you need to know about SciPy}
      It's good to know that SciPy functions are there, for when you need them.
      Things that might come up for you during this class:
      \begin{enumerate}
        \item<2 -> Need a statistical function: scipy.stats
        \item<3 -> Need to process an image: scipy.ndimage
        \item<4 -> if these things come up: scipy.io often has the functionality
                   for getting various libraries to `talk' to each other.
      \end{enumerate}
  }

  \frame{
    \frametitle{Pandas}
      Tables are important! \vspace{1cm}
  
    \onslide<2 -> {
  \begin{tabular}{|c|c|c|c|}
  \hline
  Day & Min Temp & Max Temp & Sky \\ \hline
  Monday  & 0C & 5C & Cloudy \\
  Tuesday & 1C & 3C & Overcast \\
  Wednesday & 3C & 8C & Sunny \\ \hline
  
  \end{tabular}}
  }


  \begin{frame}[fragile]
    \frametitle{Pandas}
      Pandas is for working with and processing tabular data. \vspace{1cm}
    \onslide<2 -> {
  \begin{tabular}{|c|c|c|c|}
  \hline
  Day & Min Temp & Max Temp & Sky \\ \hline
  Monday  & 0C & 5C & Cloudy \\
  Tuesday & 1C & 3C & Overcast \\
  Wednesday & 3C & 8C & Sunny \\ \hline
  
  \end{tabular}} \vspace{0.5cm}
      \begin{enumerate}
        \item<3 -> Selection (rows)
        \item<4 -> Slicing (columns)
      \end{enumerate}
  \end{frame}

  \begin{frame}[fragile]
    \frametitle{Pandas}
      Pandas is for working with and processing tabular data. \vspace{1cm}
   
  \begin{tabular}{|c|c|c|c|}
  \hline
  Day & Min Temp & Max Temp & Sky \\ \hline
  Monday  & 0C & 5C & Cloudy \\
  Tuesday & 1C & 3C & Overcast \\
  Wednesday & 3C & 8C & Sunny \\ \hline
  
  \end{tabular} \vspace{0.5cm}
      \begin{enumerate}
        \item<2 -> Reduce
        \item<3 -> Aggregate
      \end{enumerate}
  \end{frame}

  \begin{frame}[fragile]
    \frametitle{Pandas}
      Pandas is for working with and processing tabular data. \vspace{1cm}
   
  \begin{tabular}{|c|c|c|c|}
  \hline
  Day & Min Temp & Max Temp & Sky \\ \hline
  Monday  & 0C & 5C & Cloudy \\
  Tuesday & 1C & 3C & Overcast \\
  Wednesday & 3C & 8C & Sunny \\ \hline
  
  \end{tabular} \vspace{0.5cm}
      \begin{enumerate}
        \item<2 -> Map
        \item<3 -> may increase or decrease number of columns!
      \end{enumerate}
  \end{frame}

  \begin{frame}[fragile]
    \frametitle{Pandas}
      Pandas is for working with and processing tabular data. \vspace{1cm}
   
  \begin{tabular}{|c|c|c|c|}
  \hline
  Day & Min Temp & Max Temp & Sky \\ \hline
  Sunday  & 0C & 8C & Sunny \\
  Monday  & 0C & 5C & Cloudy \\
  Tuesday & 1C & 3C & Overcast \\
  Wednesday & 3C & 8C & Sunny \\ \hline
  
  \end{tabular} \vspace{0.5cm}
      \begin{enumerate}
        \item<2 -> GroupBy
      \end{enumerate}
  \end{frame}

  \frame{
    \frametitle{High-level view}
      Many of the same operations you'd see in SQL (but be warned!).
      \begin{enumerate}
        \item<2 -> Join/Merge
        \item<3 -> Intersection/Union/Concat
        \item<4 -> `index' in pandas is like `key' in SQL
      \end{enumerate}
  }

  \frame{
    \frametitle{To the Noteboo}
    What the title says
  }


%%%%%%%%%%%%%%%%%%%%%%%%%%%%%%%%%%%%%%%% 
%%% Conclusion
%%%%%%%%%%%%%%%%%%%%%%%%%%%%%%%%%%%%%%%% 

  \frame{
    \frametitle{Thanks for your time!}
  }

\end{document}
