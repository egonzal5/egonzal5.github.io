\documentclass{beamer}

\usepackage{amssymb}
\usepackage{fancyvrb}
\usepackage{stmaryrd}
\usepackage{graphicx}
\usefonttheme{serif}


\newcommand{\Nat}{\mathbb{N}}

\title{Getting Data}%\texorpdfstring{$\mathbb{N}$}}
\subtitle{Gottem}
\date{Febuary 8\textsuperscript{st}, 2021}

\usetheme{jmct}

\usepackage{calc}

\newcommand{\textover}[3][l]{%
 % #1 is the alignment, default l
 % #2 is the text to be printed
 % #3 is the text for setting the width
 \makebox[\widthof{#3}][#1]{#2}%
 }

\newcommand{\blueit}[1]{%
  {\color{dark-lucid-blue}#1}%
}
\newcommand{\blueite}[1]{%
  \blueit{\emph{#1}}%
}


\newcommand{\myquote}[3]{
  ``#1''
  \vspace{3pt}
  \hrule
  \begin{flushright}
  --- \blueit{\emph{#2}}, \emph{#3}
  \end{flushright}
}

\begin{document}
	\frame {
		\titlepage
	}

%%%%%%%%%%%%%%%%%%%%%%%%%%%%%%%%%%%%%%%% 
%%% Intro
%%%%%%%%%%%%%%%%%%%%%%%%%%%%%%%%%%%%%%%% 

  \frame{
    \frametitle{This Lecture}
    Getting some data.
  }

  \frame{
    \frametitle{Before we start...}
      \begin{enumerate}
        \item<2 - 3> Assignment later this week.
        \item<3 - 3> Working Together, part 2
      \end{enumerate}
  }

  \frame{
    \frametitle{Assignment}
      \begin{enumerate}
        \item<2 -> Should go live Wednesday evening (i.e. after the lecture)
        \item<3 -> You should \emph{absolutely} not wait until the last minute.
      \end{enumerate}
  }

  \frame{
    \frametitle{Working Together}
      \onslide<2>{With another assignment being release, some more thoughts on working together for 320. These thoughts only apply to 320.}
    }


%%%%%%%%%%%%%%%%%%%%%%%%%%%%%%%%%%%%%%%% 
%%% Data
%%%%%%%%%%%%%%%%%%%%%%%%%%%%%%%%%%%%%%%% 

  \frame{
    \frametitle{Getting Data!}
    Last week we talked about \emph{data}. Great fine, bit whoop. We want to actually \blueit{use}
    data!
  }

  \frame{
    \frametitle{Python to the rescue}
    \onslide<2->{There's lots of data on the internet.}
    \onslide<3->{Some of it is useful.}
    \onslide<4->{Even less of it is SFW.}
    \onslide<5->{Luckily, we can write programs that let us get this data.}
  }

  \frame{
    \frametitle{Python to the rescue}
      How do we do this? A few things to keep in mind:
      \begin{enumerate}
        \item<2 -> Every website does their own thing, even if they claim to meet a standard.
        \item<3 -> You're going to have to get comfortable with \blueit{exploring} what you get back.
        \item<4 -> All hope is not lost, there are some common things that will help.
      \end{enumerate}
  }

  \frame{
    \frametitle{A light in the darkness}
      \begin{enumerate}
        \item<2 -> Learning how JSON works will pay dividends (eventually you won't even think about it)
        \item<3 -> CSV is crucial and will almost certainly come up.
        \item<4 -> Learning some basic HTML will help, but understand that few sites produce compliant HTML
        \item<5 -> GraphQL is the new kid on the block, unclear how popular it will be (maybe huge!)
      \end{enumerate}
  }

  \frame{
    \frametitle{Some encouragement}
      I would be doing you a disservice if I forced you to learn the details of these formats.
      \begin{enumerate}
        \item<2 -> If you go in thinking that a website has definitely followed the standard, you're only producing tears.
        \item<3 -> Use an interactive environment (REPL, Jupyter Notebook, etc.)
      \end{enumerate}
  }

  \frame{
    \frametitle{To the Notebook!}
      What the title says.
  }


%%%%%%%%%%%%%%%%%%%%%%%%%%%%%%%%%%%%%%%% 
%%% Conclusion
%%%%%%%%%%%%%%%%%%%%%%%%%%%%%%%%%%%%%%%% 

  \frame{
    \frametitle{What did we learn?}
      \begin{enumerate}
        \item<2 -> Be careful with passwords!
        \item<3 -> Look up the API docs for the website you're trying to use
        \item<4 -> Realize the docs are bad
        \item<5 -> Go through the stages of acceptance
        \item<6 -> Explore/play with what you get and press on
      \end{enumerate}
  }

  \frame{
    \frametitle{What didn't we learn?}
      \begin{enumerate}
        \item<2 -> Beautiful Soup (HTML)
        \item<3 -> CSV
        \item<4 -> Manipulating JSON into other formats
        \item<5 -> For all of these: read the docs of the libraries!
      \end{enumerate}
  }

  \frame{
    \frametitle{Thanks for your time!}
  }

\end{document}
