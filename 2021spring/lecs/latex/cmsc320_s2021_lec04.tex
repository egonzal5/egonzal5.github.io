\documentclass{beamer}

\usepackage{amssymb}
\usepackage{fancyvrb}
\usepackage{stmaryrd}
\usepackage{graphicx}
\usefonttheme{serif}


\newcommand{\Nat}{\mathbb{N}}

\title{Data Science}%\texorpdfstring{$\mathbb{N}$}}
\subtitle{CMSC 320}
\date{Febuary 3\textsuperscript{st}, 2021}

\usetheme{jmct}

\usepackage{calc}

\newcommand{\textover}[3][l]{%
 % #1 is the alignment, default l
 % #2 is the text to be printed
 % #3 is the text for setting the width
 \makebox[\widthof{#3}][#1]{#2}%
 }

\newcommand{\blueit}[1]{%
  {\color{dark-lucid-blue}#1}%
}
\newcommand{\blueite}[1]{%
  \blueit{\emph{#1}}%
}


\newcommand{\myquote}[3]{
  ``#1''
  \vspace{3pt}
  \hrule
  \begin{flushright}
  --- \blueit{\emph{#2}}, \emph{#3}
  \end{flushright}
}

\begin{document}
	\frame {
		\titlepage
	}

%%%%%%%%%%%%%%%%%%%%%%%%%%%%%%%%%%%%%%%% 
%%% Intro
%%%%%%%%%%%%%%%%%%%%%%%%%%%%%%%%%%%%%%%% 

  \frame{
    \frametitle{This Lecture}
    Getting some data.
  }

  \frame{
    \frametitle{Before we start...}
      \begin{enumerate}
        \item<2 - 5> Queuing for office hours.
        \item<3 - 5> Working together.
        \item<4 - 5> Email.
        \item<5 - 5> Accommodation Letters.
      \end{enumerate}
  }

  \frame{
    \frametitle{Queuing}
      \begin{enumerate}
        \item<2 -> Thank you for your feedback!
        \item<3 -> After discussing with students/TAs, we're going to use discord, not Quuly.
        \item<4 -> Allow me to demonstrate.
        \item<5 -> If this does not work for you, let us know! We want to be as flexible as possible.
      \end{enumerate}
  }

  \frame{
    \frametitle{Working Together}
      \onslide<2>{Some thoughts on working together for 320. These thoughts only apply to 320.}
    }

  \frame{
    \frametitle{Email}
    \onslide<2>{I get a lot of email}
    \onslide<3>{\begin{itemize}
      \item Never feel shy to email again.
      \item If you're still shy, email a TA and they'll reach me.
    \end{itemize}
    }
  }

  \frame{
    \frametitle{Accommodation Letters}
      \begin{enumerate}
        \item<2 -> For some reason, my ads portal account is borked.
        \item<3 -> While IT is on it, I've lost my patience. Moving to Plan B.
        \item<4 -> If you email me with your letter (i.e. not through the ADS portal) I will respond with affirmation that I have seen your letter.
        \item<5 -> If you would like further assurance, I will print it out, sign it, and scan it back for you.
        \item<6 -> I promise this is not my ideal situation, I'm sorry that it's affected how quickly I can turn around these letters.
      \end{enumerate}
  }


%%%%%%%%%%%%%%%%%%%%%%%%%%%%%%%%%%%%%%%% 
%%% Data
%%%%%%%%%%%%%%%%%%%%%%%%%%%%%%%%%%%%%%%% 

  \frame{
    \frametitle{Data!}
    The reading teased the notion of 4 `kinds' of data (in roughly two `types'):
    \begin{itemize}
      \item<2-> Nominal (Categorical)
      \item<3-> Ordinal (Categorical)
      \item<4-> Interval (Numerical)
      \item<5-> Ratio (Numerical)
    \end{itemize}

  }

  \frame{
    \frametitle{Categorical Data: Nominal}
    \begin{itemize}
      \item<2-> Think `finite set'
      \item<3-> Marital status, soda flavor, etc.
      \item<4-> Comparison is difficult and nonsensical
    \end{itemize}
  }

  \frame{
    \frametitle{Categorical Data: Ordinal}
    \begin{itemize}
      \item<2-> Like Nominal data, Ordinal data describes \blueit{classes} or \blueit{states} of things...
      \item<3-> But we can provide an \blueit{order}
      \item<4-> The lecturer of this class is \{boring, neutral, exciting\}
      \item<5-> We have an order but not a mathematical way to measure \blueit{distance}
    \end{itemize}
  }


  \frame{
    \frametitle{Numerical Data: Interval}
    \begin{itemize}
      \item<2-> Think: Dates, year in school (i.e. grade level), temperature.
      \item<3-> We have ordering \blueit{and} distance.
      \item<4-> What don't we have?
    \end{itemize}
  }

  \frame{
    \frametitle{Numerical Data: Ratio}
    \begin{itemize}
      \item<2-> Everything Interval has, but with a meaningful \blueit{zero}
      \item<3-> Ratios are meaningful (hence the name)
      \item<4-> Money, distance, volume, etc.
    \end{itemize}
  }

  \frame{
    \frametitle{From data to data \emph{representation}}
    \onslide<2->{Data structures are important!}
    \onslide<3->{They guide you by limiting the number of appropriate operations}
    \begin{itemize}
      \item<4-> What are the appropriate operations for an array?
      \item<5-> Index, slice, map, reduce, etc.
      \item<6-> What dataset would be appropriate to represent as an array?
      \item<7-> In what ways could we combine two arrays?
    \end{itemize}
  }

  \frame{
    \frametitle{From data to data \emph{representation}}
    \onslide<2->{What about multi-dimensional arrays?}
  }

  \frame{
    \frametitle{From data to data \emph{representation}}
    \onslide<2->{What about $\Nat$-dimensional arrays (i.e. higher-dimensional matrices)}
    \begin{itemize}
      \item<3-> This is where Linear Algebra starts to come in handy!
    \end{itemize}
  
  }

  \frame{
    \frametitle{From data to data \emph{representation}}
    \onslide<2->{What about...}
    \begin{itemize}
      \item<3-> Sets?
      \item<4-> Maps (a.k.a Dictionaries)?
      \item<5-> Tables?
      \item<6-> Trees?
      \item<7-> Graphs?
    \end{itemize}
  
  }

  \frame{
    \frametitle{Let's get some data!}
    \onslide<2->{To the REPL!}
  }

%%%%%%%%%%%%%%%%%%%%%%%%%%%%%%%%%%%%%%%% 
%%% Conclusion
%%%%%%%%%%%%%%%%%%%%%%%%%%%%%%%%%%%%%%%% 

  \frame{
    \frametitle{Any Questions?}
  }

  \frame{
    \frametitle{Thanks for your time!}
  }

\end{document}
